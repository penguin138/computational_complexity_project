\documentclass{article}

\usepackage[english, russian]{babel}
\usepackage[utf8]{inputenc}
\usepackage{amsthm}
\usepackage{amsmath}
\title{Дополнительные задачи к контрольной №1}
\author{Торунова Анастасия. 394 группа}

\begin{document}
\maketitle
\section*{Задача 3.}
\subsection*{к)}
$\mathsf{BIPARTITESUBGRAPH}=\{ (G,k) |$ в графе $G$ найдется двудольный подграф, содержащий хотя бы $k$ вершин $ \} $

Сначала покажем, что данный язык лежит в $\mathbf{NP}$. Воспользуемся сертификатным определением. Действительно, тогда сертификатом будут вершины подграфа, разделенные на 2 доли. Тогда верификатор проверяет, что в каждой доле вершин $\geq k/2$ (за $O(|V|)$), затем то, что вершины в доле не соединены друг с другом ребрами (за $O(|V|^2)$).Он это делает за полином, поэтому $\mathsf{BIPARTITESUBGRAPH} \in \mathbf{NP}$.

Теперь покажем, что $\mathsf{BIPARTITESUBGRAPH}$\ $\mathbf{NP}$-трудный. Воспользуемся $\mathbf{NP}$-полнотой задачи $\mathsf{INDSET} =  \{ (G,k)|$ в графе $G$ найдется независимое множество размера $\geq k$ $\}$.
Сведем $\mathsf{INDSET}$ к $\mathsf{BIPARTITESUBGRAPH}$. Рассмотрим вход для задачи $\mathsf{INDSET}$: $(G,k)$.
Скопируем граф G два раза,получим его копии $G_1$ и $G_2$ и проведем между ними все ребра. Теперь докажем, что в полученном графе $G'$ есть двудольный подграф размера $\geq 2k$ $\Leftrightarrow$  в $G$ есть независимое множество размера $\geq k$.

Пусть в $G'$ есть двудольный подграф размера $\geq 2k$. Тогда размер какой-то  из долей $\geq k$, а она является независимым множеством по определению двудольного графа. Покажем, что она обязательно лежит в одной из копий $G$.Пусть не так, но тогда она содержит одновременно вершину из $G_1$ и вершину из $G_2$, но по построению $G'$ между ними проведено ребро, что противоречит тому, что доля -- независимое множество. Итак мы нашли независимое множество размера $\geq k$, которое лежит целиком в одной из копий $G$, значит мы нашли это множество в $G$.

Пусть теперь в $G$ есть независимое множество $H$ размера $\geq k$. Тогда получаем, что в $G_1$ и $G_2$ есть копии этого множества -- $H_1$ и $H_2$ соответственно. Рассмотрим граф, образованный вершинами $H_1$ и $H_2$. Понятно, что $H_1$ и $H_2$ образуют его доли, а его размер $\geq 2k$. Значит мы нашли нужный подграф в $G'$.

Таким образом мы построили сведение $\mathsf{INDSET}$ к $\mathsf{BIPARTITESUBGRAPH}$. Оно полиномиальное, так как копирование графа происходит за $O(|G|)$, а проведение дополнительных ребер за $O(|V|^2)$.
Получаем, что $\mathsf{BIPARTITESUBGRAPH}$ --$\mathbf{NP}$ -трудный, а значит, по доказанному в начале, $NP$-полный.
\end{document} 
